\section*{Appendix}
\subsection{Analysis of Data}
For each measurement, data was taken from the oscilloscope and stored in raw text (csv) files labelled with the varied parameter and scintillator crystal used in a A vs B fashion. Each measurement produced 7 datafiles, corresponding to different recorded information from the programmed oscilloscope. These are :
\begin{enumerate}
\item Maximum energy per event of the left scintillator detector
\item Maximum energy per event of the right scintillator detector
\item The relative arrival time between coincidence events
\item (Not used) Width of left NINO signal
\item (Not used) Width of right NINO signal
\item (Not used)
\item The number of events per period in the left scintillator detector
\item The number of events per period in the right scintillator detector
\end{enumerate}
Where files 1 and 2 are the maximum energy per event, files 7,8 are the events per period and file 3 is the relative arrival time. The remaining two files correspond to the widths of the NINO signals. These are not used.

To reach the processed data attached to this research paper, the following tasks were performed upon the raw data.

\begin{enumerate}
\item Load grouped data into memory
\item Searches for and fits to Gaussian peaks in energy data
\item Searches for events which record only two gamma ray photons
\item Select events matching photopeak and two event criteria
\item Use above to select for true gamma-gamma events only
\item Fit to delay peak. Determine error in fit using BCA bootstrap.
\item Return results
\end{enumerate}

\subsection{Data, Tables and Images}
All processed data can be found \href{https://github.com/marksbrown/doipaper/tree/master/processeddata}{here} for each set of parameter - scintillator crystals, configurations and shapes are combined and plotted \& tabulated in the following notebooks :
\begin{itemize}
\item \href{http://nbviewer.ipython.org/github/marksbrown/doipaper/blob/master/notebooks/Generate\%20Images.ipynb}{Generated Images}
\item \href{http://nbviewer.ipython.org/github/marksbrown/doipaper/blob/master/notebooks/Generate\%20Tables.ipynb}{Generated Tables}
\end{itemize}
Results not presented in the paper, which may prove useful are available :
\begin{itemize}
\item \href{http://nbviewer.ipython.org/github/marksbrown/doipaper/blob/master/notebooks/Generate\%20Additional\%20Images.ipynb}{Generated Additional Images}
\item \href{http://nbviewer.ipython.org/github/marksbrown/doipaper/blob/master/notebooks/Generate\%20Additional\%20Tables.ipynb}{Generated Additional Tables}
\end{itemize}

The raw data, along with notebooks showing the fits and parameters of every measurement taken can be found at \href{http://www.ee.ucl.ac.uk/~mbrown/}{http://www.ee.ucl.ac.uk/~mbrown/}.
